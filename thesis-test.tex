\documentclass[master]{suribt}
%\documentclass[oneside]{suribt}% 本文が * ページ以下のときに (掲示に注意)
\title{特許検索における質問意図の曖昧化}
%\titlewidth{}% タイトル幅 (指定するときは単位つきで)
\author{胡瀚林}
\eauthor{HANLIN HU}% Copyright 表示で使われる
\studentid{48-156229}
\supervisor{中川裕志 教授}% 1 つ引数をとる (役職まで含めて書く)
%\supervisor{指導教員名 役職 \and 指導教員名 役職}% 複数教員の場合,\and でつなげる
\handin{2017}{01}% 提出月. 2 つ (年, 月) 引数をとる
%\keywords{キーワード1, キーワード2} % 概要の下に表示される

\begin{document}
\maketitle%%%%%%%%%%%%%%%%%%% タイトル %%%%

\frontmatter% ここから前文
\begin{abstract}%%%%%%%%%%%%% 概要 %%%%%%%%
 企業が特許を取る前に,類似な特許が既に存在するかを確かめるために特許データベースを検索する必要がある.
 しかし,検索の質問から企業秘密が漏洩する可能性がある.
 ウェブテキスト検索の質問からユーザーの興味を守る手法が多数存在している.
 その中真の質問と同時にダミー質問を提出する手法が一番効率的、現実的である.
 一般的なウェブテキスト検索と違い,特許データベース検索は長い検索質問を用い,検索の精度と再現率を重視しているため、既存手法を直接特許検索に適用することはできない.
 本発表では既存手法を破られる攻手法を提案し、その攻撃手法に対応できるダミー質問生成システムを提案する.
 \end{abstract}

 \tableofcontents%%%%%%%%%%%%% 目次 %%%%%%%%

 \mainmatter% ここから本文 %%% 本文 %%%%%%%%
 \chapter{はじめに}
 \chapter{特許}
 \section{特許文章}
 \section{国際特許分類}
 \section{特許検索}
 \chapter{曖昧化検索}
 \section{否認可能検索を利用したプライバシー保護\cite{providing2009}}
 \section{質問者のプライバシーを保護する質問加工法\cite{embellishing2010}}
 \section{質問意図を曖昧化するキーワード検索\cite{masking2014}}
 \chapter{語意分析}
 \section{tf-idf}
 \section{潜在意味解析}
 \section{潜在的ディリクレ配分法}
 \chapter{プライバシー分析(攻撃手法)}
 \section{メイントピック攻撃}
 \section{類似度攻撃\cite{simattack2016}(事前情報あり)}
 \section{類似度攻撃2(事前情報なし)}
 \chapter{質問曖昧化(提案手法)}
 \section{単語ベクトル}
 \section{質問曖昧化}
 \chapter{データベース分割}
 \chapter{実験}
 \section{データベース}
 \section{tfidf vs lda vs lsa}
 \section{データベース分割}
 \section{検索結果分析(真の質問が当たられる確率 vs ダミー質問と真の質問の検索結果の類似度)}
 \chapter{おわりに}

 \backmatter% ここから後付
 \chapter{謝辞}%%%%%%%%%%%%%%% 謝辞 %%%%%%%

  %\begin{thebibliography}{}%%%% 参考文献 %%%
  % \bibitem{}
  %\end{thebibliography}
  \bibliographystyle{tieice}%           BibTeX を使う場合
  \bibliography{thesis}% BibTeX を使う場合

  \appendix% ここから付録 %%%%% 付録 %%%%%%%
  \chapter{}
  \end{document}
