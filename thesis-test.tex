\documentclass[master]{suribt}
%\documentclass[oneside]{suribt}% 本文が * ページ以下のときに (掲示に注意)
\title{特許検索における質問意図の曖昧化}
%\titlewidth{}% タイトル幅 (指定するときは単位つきで)
\author{胡瀚林}
\eauthor{HANLIN HU}% Copyright 表示で使われる
\studentid{48-156229}
\supervisor{中川裕志 教授}% 1 つ引数をとる (役職まで含めて書く)
%\supervisor{指導教員名 役職 \and 指導教員名 役職}% 複数教員の場合,\and でつなげる
\handin{2017}{01}% 提出月. 2 つ (年, 月) 引数をとる
%\keywords{キーワード1, キーワード2} % 概要の下に表示される

\begin{document}
\maketitle%%%%%%%%%%%%%%%%%%% タイトル %%%%

\frontmatter% ここから前文
\begin{abstract}%%%%%%%%%%%%% 概要 %%%%%%%%
 ここに概要を書く.
 \end{abstract}

 \tableofcontents%%%%%%%%%%%%% 目次 %%%%%%%%

 \mainmatter% ここから本文 %%% 本文 %%%%%%%%
 \chapter{はじめに}
 \chapter{特許}
 \section{特許文章}
 \section{特許検索}
 \chapter{曖昧化検索}
 \section{}
 \section{}
 \section{}
 \chapter{語意分析}
 \section{tf-idf}
 \section{潜在意味解析}
 \section{潜在的ディリクレ配分法}
 \chapter{おわりに}
 \section{特許検索}
 \section{特許検索}
 \section{特許検索}
 \section{特許検索}

 \backmatter% ここから後付
 \chapter{謝辞}%%%%%%%%%%%%%%% 謝辞 %%%%%%%

  %\begin{thebibliography}{}%%%% 参考文献 %%%
  % \bibitem{}
  %\end{thebibliography}
  \bibliographystyle{thesis}%           BibTeX を使う場合
  %\bibliography{.bib ファイル名}% BibTeX を使う場合

  \appendix% ここから付録 %%%%% 付録 %%%%%%%
  \chapter{}
  \end{document}
