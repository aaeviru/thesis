\documentclass[master]{suribt}
\usepackage{graphicx}
\usepackage{algorithm}
\usepackage{algorithmicx}
\usepackage{algpseudocode}
\renewcommand{\algorithmicrequire}{\textbf{Input:}}
\renewcommand{\algorithmicensure}{\textbf{Output:}}
\newcommand{\argmax}{\mathop{\rm argmax}\limits}
\newcommand{\tabincell}[2]{\begin{tabular}{@{}#1@{}}#2\end{tabular}}
%\documentclass[oneside]{suribt}% 本文が * ページ以下のときに (掲示に注意)
\title{特許検索における質問意図の曖昧化}
%\titlewidth{}% タイトル幅 (指定するときは単位つきで)
\author{胡瀚林}
\eauthor{HANLIN HU}% Copyright 表示で使われる
\studentid{48-156229}
\supervisor{中川裕志 教授}% 1 つ引数をとる (役職まで含めて書く)
%\supervisor{指導教員名 役職 \and 指導教員名 役職}% 複数教員の場合,\and でつなげる
\handin{2017}{01}% 提出月. 2 つ (年, 月) 引数をとる
%\keywords{キーワード1, キーワード2} % 概要の下に表示される

\begin{document}
\maketitle%%%%%%%%%%%%%%%%%%% タイトル %%%%

\frontmatter% ここから前文
\begin{abstract}%%%%%%%%%%%%% 概要 %%%%%%%%
 企業が特許を取る前に,類似な特許が既に存在するかを確かめるために特許データベースを検索する必要がある.
 しかし,検索の質問から企業秘密が漏洩する可能性がある.
 ウェブテキスト検索の質問からユーザーの検索意図を守る手法が多数存在している.
 その中真の質問と同時にダミー質問を提出する質問曖昧化手法が一番効率的、現実的である.
 本稿では特許検索における既存な質問曖昧化手法\cite{providing2009,embellishing2010,masking2014}を実装し,類似度攻撃\cite{simattack2016}で既存手法の安全性を評価した.

 また、類似度攻撃\cite{simattack2016}を含め、多くの既存な質問曖昧化に対する攻撃手法は攻撃者が質問者の事前情報を持つと仮定する.
 本稿では事前情報なしの攻撃手法を提案し,その攻撃手法に対応できる既存な質問曖昧化の改良と新たな質問曖昧化手法を提案する.
\end{abstract}

 \tableofcontents%%%%%%%%%%%%% 目次 %%%%%%%%

 \mainmatter% ここから本文 %%% 本文 %%%%%%%%
 \chapter{はじめに}
 テキスト検索をするとき,検索質問をサーバー側に渡さなければならない.
 しかし,検索質問からユーザーの情報が漏洩する危険があることがAOL事件\cite{_face_2006}より証明された.
 特に特許検索の場合は検索質問が研究開発動向など企業秘密を含んでいるため,一般的なウェブ検索の質問者より質問のプライバシー問題を重視している.

 \chapter{特許}
 特許検索質問のプライバシーを保護する手法を説明する前に特許検索と特許そのものを簡単に紹介する必要がある.
 特許法第1条には,「この法律は、発明の保護及び利用を図ることにより、発明を奨励し、もつて産業の発達に寄与することを目的とする」とある.
 特許制度は,発明者には一定期間,一定の条件のもとに特許権という独占的な権利を与えて発明の保護を図る一方,
 その発明を公開して利用を図ることにより新しい技術を人類共通の財産としていくことを定めて,
 これにより技術の進歩を促進し,産業の発達に寄与しようというものである.\cite{https://www.jpo.go.jp/seido/s_tokkyo/chizai04.htm}
 特許を取るには以下の条件を満たさなければならない:
 新規性:公知の発明と同様の発明は特許を受けることができない;
 進歩性:先行技術に基づいて容易に発明をすることができる発明は特許を受けることができない.
 単一性:発明の単一性の要件を満たさない二以上の発明は一つの願書で出願することができない.

 特許を受けようとする発明を特定するために特許請求の範囲を記載する必要がある.
 \begin{figure}
  \hspace*{-2cm}
  \begin{tabular}{cc}
  \includegraphics[width=0.55\textwidth,natwidth=500,natheight=1100]{ex1-1.pdf}
  \includegraphics[width=0.55\textwidth,natwidth=500,natheight=1100]{ex1-2.pdf}
  \end{tabular}
  \caption{特許文章例}
  \label{fig:exp}
 \end{figure}

 図\ref{fig:exp}で表した例のように,特許の請求項は特定の書き方がある.
 誤解を招かないように技術用語は、学術用語を用いる.
 また,一般的な文章は単語をなるべく重複しないようにする一方,特許文章は単語を全体を通じて統一して使用する.
 \section{特許分類}
 特許の一つ特徴は全ての特許が人の手によって分類されている.
 特許分類を用いることより検索する特許文章が減り,似たようなキーワードを含むが分類が違う特許文章を排除することができる.
 今最も使われている特許分類が世界知的所有権機関(WIPO)による管理されている国際特許分類(IPC)である.
 国際特許分類は階層構造であり,一番上の階層はAからHまでの8個のセクションである.
 セクション以下は\ref{tab:IPC}に表したように四つの階層に分類されている.
 
 \begin{table}[!hbp]
 \center
 \begin{tabular}{cc}
    セクション:A & 健康および娯楽 \\
    サブセクション : 61 & 医学または獣医学:衛生学 \\
    クラス: C & 歯科:口腔または歯科衛生 \\
    メイングループ:5 & 歯の充填または被覆 \\
    サブグループ:08 & 歯冠:その製造;口中での歯冠固定 \\
 \end{tabular}
 \caption{国際特許分類例:A61C 5/08}
 \end{table}

 \section{特許検索}
 \begin{table}[!hbp]
 \center
 \begin{tabular}{|c|c|c|}
 \noalign{\hrule height 1pt}
 検索タイプー & 検索対象(specification) & 検索目的 \\
 \hline
 \tabincell{c}{技術水準調査\\(State of the Art Search)} & イデア & 自分の発明に関連する背景知識を得る \\
 \tabincell{c}{新規性調査\\(Novelty Search)} & 特許文章 & 特許登録の可能性を判断する \\
 \tabincell{c}{侵害調査\\(Infringement Search)} &  \tabincell{c}{商品と\\商品に関連する技術} & 権利侵害とならないかを判断する \\
 \noalign{\hrule height 1pt}
 \end{tabular}
 \caption{特許検索タイプー}
 \end{table}


 \begin{table}[!hbp]
 \center
 \begin{tabular}{|c|c|}
 \hline
 符号 & 意味 \\
 \hline
 $N$ & 辞書中の単語の数 \\
 $W = \{1,2,3, \dots ,N\} $ & 単語集合 \\
 $M$ & コーパス中の文書の数 \\
 $D = \{1,2,3, \dots ,M\}$ & 文章集合 \\
 $K$ & トピック数 \\
 $T = \{1,2,3, \dots ,K\}$ & トピック集合 \\
 $\ell_i = \{t_1,t_2,\dots,K\} $ & 単語$i$のトピックベクトル \\
 $\ell$ & 質問のトピックベクトル \\
 \hline
 \end{tabular}
 \caption{表記法}
 \end{table}

 \chapter{曖昧化検索}

 \section{否認可能検索を利用したプライバシー保護\cite{providing2009}}
 \section{質問者のプライバシーを保護する質問加工法\cite{embellishing2010}}
 \section{質問意図を曖昧化するキーワード検索\cite{masking2014}}
 \begin{algorithm}
 \caption{HDGA(On Masking Topical Intent in Keyword Search)}
 \begin{algorithmic}[1]
  \Require 質問:$q_1$
  \State $Q = \{q_1\}\delta_{q_1} = \argmax_{t \in T} Pr[t|q_1]$
  \ForAll {$t \in T \setminus \{\delta_{q_1}\}$}
  \State $e_t = h(\delta_{q_1}||t||s)$
  \EndFor
  \State $T_D = \{t^1_{q1},t^2_{q1}, \dots , t^2_{q1} | \forall t_1 \ \in T_D , \forall t_2 \ \in T \setminus T_D, e_{t_1} > e_{t_2} \}$
  \ForAll {$t \in T_D $}
  \While { $ \argmax_{t \in T} Pr[t|q'] \neq t$}
  \State randomly select $|q_1|$ keywords for $t$ based on $Pr[w|t]$, to form a dummy query $q'$
  \EndWhile
  \State $Q = Q \cup \{q'\}$
  \EndFor 
  \State Shuffle queries in Q
  \Return $Q$
 \end{algorithmic}
 \end{algorithm}

 \chapter{語意分析}
 \section{tf-idf}
 \section{潜在意味解析}
 \section{潜在的ディリクレ配分法}
 \chapter{プライバシー分析(攻撃手法)}
 \section{メイントピック攻撃}
 \begin{algorithm}
 \caption{メイントピック攻撃}
 \begin{algorithmic}[1]
  \Require 質問:$q=\{t_i\},$単語のトピックベクトル集合$L=\{\ell_i\}$
  \State $R=\phi, \, \ell=0$
  \State $\ell=\sum_{t_i \in Q}\ell_{t_i}$
  \State $maintopic = \argmax_j \ell[j]$
  \ForAll {$bk_k \in q $}
  \State $R=R \cup \{\max_{t_i}l_{t_i}[maintopic]\}$
  \EndFor \\
  \Return $R$
 \end{algorithmic}
 \end{algorithm}

 \section{類似度攻撃\cite{simattack2016}(事前情報あり)}
 \begin{algorithm}
 \caption{類似攻撃}
 \begin{algorithmic}[1]
  \Require 質問集合:$Q=\{q^r_i\, | i \in \{1,2,3,4\},r \in \{1,2, \dots ,R\}\},$単語のトピックベクトル集合$L=\{\ell_i\}$
  \State $p_i= q^1_i \, \, i \in \{1,2,3,4\} , result = \phi $
  \For {$r = 2,3 \dots ,R $}
   \For {$i = 1,2,3,4$}
    \State $j = \argmax_j \frac{p_i \cdot q^r_j}{|p_i||q^r_j|}$
    \State $d_i = \frac{p_i \cdot q^r_j}{|p_i||q^r_j|}$
    \State $temp_i = \frac{1}{r}(p_i(r-1) + q_j)$
   \EndFor
   \For {$i = 1,2,3,4$}
    \State $p_i = temp_i $
   \EndFor
   \State $result =result \cup \{\argmax_{i}d_i\}$
  \EndFor \\
  \Return $result$
 \end{algorithmic}
 \end{algorithm}

 \section{類似度攻撃2(事前情報なし)}


 \chapter{質問曖昧化(提案手法)}
 \section{単語ベクトル}
 \section{質問曖昧化}
 \chapter{データベース分割}
 特許分類を用いることにより特許データベースを分割することができる.
 分割したデータベース各々に対して同じような信憑性を持つ質問を提出すると真に検索したいデータベースを隠すことができると考えられる.

 \chapter{実験}
 \section{データベース}
 \section{tfidf vs lda vs lsa}
 \section{データベース分割}

 \section{検索結果分析(真の質問が当たられる確率 vs ダミー質問と真の質問の検索結果の類似度)}
 \chapter{おわりに}

 \backmatter% ここから後付
 \chapter{謝辞}%%%%%%%%%%%%%%% 謝辞 %%%%%%%

  %\begin{thebibliography}{}%%%% 参考文献 %%%
  % \bibitem{}
  %\end{thebibliography}
  \bibliographystyle{tieice}%           BibTeX を使う場合
  \bibliography{thesis}% BibTeX を使う場合

  \appendix% ここから付録 %%%%% 付録 %%%%%%%
  \chapter{}
  \end{document}
