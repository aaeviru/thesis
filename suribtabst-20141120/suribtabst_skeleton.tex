\documentclass[papersize]{suribtabst}

% 必要なパッケージ又はマクロはこの辺りに定義
% 下記はサンプルで,常に読み込まねばならないわけではない.
\usepackage{graphicx}
\usepackage{amsmath}
\usepackage{amssymb}
\usepackage{amsthm}
\usepackage{amsfonts}
\usepackage{latexsym}

% 論文タイトル:日本語だけの場合
\title{数理情報学専攻における修論執筆に関する研究}
% 論文タイトル:英語の場合は日本語タイトルを併記
%\title{Writing a Master's Thesis in Mathematical Engineering Department\\ \vspace{0.4em}(数理情報学専攻における修論執筆に関する研究)}
 
%\titlewidth{}% タイトル幅 (指定するときは単位つきで)
\author{胡 瀚林}
\eauthor{HANLIN HU}% Copyright 表示で使われる
\studentid{48-156229}
\supervisor{中川 裕志 教授}% 1 つ引数をとる (役職まで含めて書く)
%\supervisor{指導教員名 役職 \and 指導教員名 役職}% 複数教員の場合,\and でつなげる
\handin{2017}{1}% 提出月. 2 つ (年, 月) 引数をとる

% 本文ここから

\begin{document}
\maketitle

\section{はじめに} % セクション名等は適宜付け替えること

数理情報学専攻の修士論文審査(会場)においては,
A4紙裏表1枚の「要約」を配付する必要がある.
そのための標準的なクラスファイル
({\texttt suribtabst.cls})が用意されており,
本スケルトンファイルは,そのクラスファイルを用いて
実際に要約を執筆する際のスタート地点となるファイルである.

必ずしもこれらのファイルを使用する義務はないが,
その出力を標準として参考にすることを推奨する.

\section{既存研究}

本ページ,以下空白.(このスケルトンにおいて)

二段組みの雰囲気を味わうためにダミー文字列を入れます:
あああああああああああああああああああああああああああああああああああああ
あああああああああああああああああああああああああああああああああああああ
あああああああああああああああああああああああああああああああああああああ
あああああああああああああああああああああああああああああああああああああ
あああああああああああああああああああああああああああああああああああああ
あああああああああああああああああああああああああああああああああああああ
あああああああああああああああああああああああああああああああああああああ

あああああああああああああああああああああああああああああああああああああ
あああああああああああああああああああああああああああああああああああああ
あああああああああああああああああああああああああああああああああああああ
あああああああああああああああああああああああああああああああああああああ
あああああああああああああああああああああああああああああああああああああ
あああああああああああああああああああああああああああああああああああああ
あああああああああああああああああああああああああああああああああああああ

あああああああああああああああああああああああああああああああああああああ
あああああああああああああああああああああああああああああああああああああ
あああああああああああああああああああああああああああああああああああああ
あああああああああああああああああああああああああああああああああああああ
あああああああああああああああああああああああああああああああああああああ
あああああああああああああああああああああああああああああああああああああ
あああああああああああああああああああああああああああああああああああああ

\clearpage

\section{本研究で提案するアルゴリズム}

以上の背景を踏まえ,本研究では以下のアルゴリズムを提案した.

% 適宜,図表等も有効に活用すること.
\begin{figure}[htbp]
\centering
\fbox{\rule{0cm}{3cm}\rule{5cm}{0cm}}
\caption{提案アルゴリズム.}
\end{figure}



% 文献リスト:

%\bibliographystyle{}%           BibTeX を使う場合
%\bibliography{}% BibTeX を使う場合

\begin{thebibliography}{99} % 普通に \bibitem を埋め込む場合
\end{thebibliography}


\end{document}